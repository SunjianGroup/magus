% Manual for the MAGUS program
% TODO: 例子中结果展示
%
\documentclass[12pt]{article}
% to add true appendix, now useless
%\usepackage{appendix}
\usepackage{graphicx}
\usepackage{color}
\usepackage{url}
\usepackage{multicol}
\usepackage[sort,super]{natbib} 
\usepackage[nottoc]{tocbibind}
\usepackage{amsmath}
\usepackage{hyperref}
\usepackage{listings}
\usepackage{CTEX}
\usepackage[most]{tcolorbox} % Required for boxes that split across pages
\usepackage{indentfirst}
\usepackage{array}
\usepackage{booktabs}
\usepackage{minted}
\usepackage{paracol} %分栏
% dirtree字体问题
\usepackage{fontspec}
\setmonofont{Source Code Pro}
% 文字背景色
\usepackage{xcolor}
\usepackage{soul}
\definecolor{Seashell}{gray}{0.95} 
\newcommand{\code}[1]{
  \begingroup
  \sethlcolor{Seashell}
  {\hl{\texttt{~#1~}}}
  \endgroup
}
\newcommand{\keyword}[1]{\texttt{#1}}
\newcommand{\file}[1]{\texttt{#1}}
\newcommand{\textshift}[1]{{\addtolength{\leftskip}{10mm}\texttt{{#1}}\par}}
\newcommand{\textshiftleft}[1]{{\addtolength{\leftskip}{-10mm}{#1}\par}}

% 页面设置
\textheight 22cm
\textwidth 16cm
\oddsidemargin 1mm
\topmargin -15mm

% 段落设置
\baselineskip=14pt
\parskip 5pt
\setlength{\parindent}{2em} % 首行缩进,如果不需要缩进使用,则在段落前使用 \noindent 

% 页眉页脚设置
\usepackage{fancyhdr}
\renewcommand{\headrulewidth}{0.4pt}
\renewcommand{\footrulewidth}{0.4pt}
\setlength{\headheight}{15pt}
\pagestyle{fancy}
\lhead{\slshape\nouppercase{\leftmark}} % controls the left corner of the header
\chead{} % controls the center of the header
\rhead{MAGUS 1.0.2} % controls the right corner of the header
\lfoot{\today} % controls the left corner of the footer
\cfoot{} % controls the center of the footer
\rfoot{Page~\thepage} % controls the right corner of the footer
\renewcommand*{\thefootnote}{\fnsymbol{footnote}}

\begin{document}

%%%%%%%%%%%%%%%%%%%%%%%%%%%%%%%%%%%%%%%%%%%%%%%%%%%%%%%%%%%%%%%%%%%%%%%%%%%%%%%%
% Title page 

\begin{titlepage}

\begin{center}

\vspace{2.0cm}
\begin{figure}[hbtp]
\centering
\includegraphics[scale=0.2]{pic/temp_logo.png}
\end{figure}
\vspace{0.5cm}
\hrule
\vspace{1.5cm}

\textbf{ 
\Large MAGUS: \vspace{0.7cm} \\
Machine learning And Graph theory assisted \vspace{0.5cm} \\
Universal structure Searcher}
\vspace{1.5cm}

\textbf{高豪, 王俊杰, 韩瑜, Dc, 孙建}
%}

\vspace{0.5cm}

\vspace{2.0cm}

\textbf{\Large 使用手册}

\textbf{Version 1.0.2, \today.}

\vspace{2.0cm}

\textcolor{blue}{\url{https://git.nju.edu.cn/gaaooh/magus}}

\end{center}

\vspace{1.0cm}

\end{titlepage}

%%%%%%%%%%%%%%%%%%%%%%%%%%%%%%%%%%%%%%%%%%%%%%%%%%%%%%%%%%%%%%%%%%%%%%%%%%%%%%%%

\newpage
\setcounter{tocdepth}{3}
\tableofcontents

\newpage
\section{安装}

\subsection{依赖}
Python >= 3.6\par
NumPy \par
SciPy \par
Scikit-learn\par
PyYAML\par
Ase>=3.19\par
Networkx==2.1\par
Spglib\par
Pandas\par
可选项:\par
Pytest >= 3.6.1: unittest\par
Xtb == 6.3: XTBCalculator\par
Mlip: MTPCalculator
\subsection{准备}
\subsubsection{设置ase的vasp接口}
\begin{itemize}
    \item [1)] 
    新建一个\file{run\_vasp.py}:
    \begin{tcolorbox}[breakable, enhanced]
    \begin{minted}[breaklines, breaksymbolleft={},]{python}
import subprocess
exitcode = subprocess.call("mpiexec.hydra /your/path/to/vasp", shell=True)
    \end{minted}
    \end{tcolorbox}
    \item [2)] 
    建立\file{mypps}目录存放vasp赝势,可以用软连接:
    \begin{verbatim}
    mypps/
    ├── potpaw
    ├── potpaw_GGA
    └── potpaw_PBE
    \end{verbatim}
    \begin{tcolorbox}[breakable, enhanced]
        \begin{minted}[breaklines, breaksymbolleft={},]{bash}
$ ln -s /your/path/PBE-5.4 mypps/potpaw_PBE
        \end{minted}
    \end{tcolorbox}
    三个子目录分别对应LDA, PW91, PBE
    也可以加入其他赝势库。
    \item [3)] 
    设置环境变量:
    \begin{tcolorbox}[breakable, enhanced]
        \begin{minted}[breaklines, breaksymbolleft={},]{bash}
$ export VASP_SCRIPT=/your/path/run_vasp.py
$ export VASP_PP_PATH=/your/path/mypps
        \end{minted}
    \end{tcolorbox}
    更多信息见:\textcolor{blue}{\url{https://wiki.fysik.dtu.dk/ase/ase/calculators/vasp.html\#module-ase.calculators.vasp}}\par
    \textbf{注意: run\_vasp.py和mypps不要放在magus目录下}
\end{itemize}
\subsubsection{机器学习包安装}
TwoShareMTPCalculator使用了修改过的mtp代码,官方版本不支持此功能.如需使用可进入magus/mtp-api目录,按其中教程安装。
详见:\textcolor{blue}{\url{https://git.nju.edu.cn/bigd4/mtp-api}}

\subsection{pip安装}
magus安装需要boost\_python,boost\_numpy.若所使用的python环境(如anaconda-3/5.0.1)的lib中有会自动探测路径,
否则需要在环境变量中给出相应路径(无需写入bashrc)
\begin{tcolorbox}[breakable, enhanced]
    \begin{minted}[breaklines, breaksymbolleft={},]{bash}
$ export MAGUS_INCLUDE_PATH=your_path_to_include_dir: your_path_to_py_include_dir
$ export MAGUS_LD_LIBRARY_PATH=your_ld_library_path
    \end{minted}
\end{tcolorbox}
如:
\begin{tcolorbox}[breakable, enhanced]
    \begin{minted}[breaklines, breaksymbolleft={},]{bash}
$ CONDA_PATH=/fs00/software/anaconda/3-5.0.1
$ export MAGUS_INCLUDE_PATH=$CONDA_PATH/include: $CONDA_PATH/include/python3.6m
$ export MAGUS_LD_LIBRARY_PATH=$CONDA_PATH/lib
    \end{minted}
\end{tcolorbox}
配置好后执行:
\begin{tcolorbox}[breakable, enhanced]
    \begin{minted}[breaklines, breaksymbolleft={},]{bash}
$ pip install -i https://repo.nju.edu.cn/repository/pypi-nju/simple magus-test --upgrade
    \end{minted}
\end{tcolorbox}

  若直接在集群安装需加入\verb|--user|参数
\subsection{source安装}
\subsubsection{代码下载}
克隆库到本地并初始化子项目:
\begin{tcolorbox}[breakable, enhanced]
    \begin{minted}[breaklines, breaksymbolleft={},]{bash}
$ git clone git@git.nju.edu.cn:gaaooh/magus.git
$ git submodule init
$ git submodule update
    \end{minted}
\end{tcolorbox}
\texttt 
或直接下载压缩包.
\subsubsection{编译GenerateNew模块}
进入\file{magus/generatenew}中,编译\file{generatenew.so}文件,并将其放入magus/magus文件夹下.
如使用集群anaconda/3-5.0.1,命令为:
\begin{tcolorbox}[breakable, enhanced]
    \begin{minted}[breaklines, breaksymbolleft={},]{bash}
$ g++ -std=c++11 -I/fs00/software/anaconda/3-5.0.1/include 
-I/fs00/software/anaconda/3-5.0.1/include/python3.6m 
-L/fs00/software/anaconda/3-5.0.1/lib -lboost_python 
-lboost_numpy -lpython3.6m main.cpp -o GenerateNew.so -shared -fPIC
    \end{minted}
\end{tcolorbox}
具体可见:
\textcolor{blue}{\url{https://git.nju.edu.cn/HanYu/generatenew}}
\subsubsection{编译lrpot模块}
进入\file{magus/lrpot}中,编译\file{lrpot.so}文件,并将其放入magus/magus文件夹下.
如使用集群anaconda/3-5.0.1,命令为:
\begin{tcolorbox}[breakable, enhanced]
    \begin{minted}[breaklines, breaksymbolleft={},]{bash}
$ g++ -std=c++11 -I/fs00/software/anaconda/3-5.0.1/include 
-I/fs00/software/anaconda/3-5.0.1/include/python3.6m 
-L/fs00/software/anaconda/3-5.0.1/lib -lboost_python 
-lboost_numpy -lpython3.6m lrpot.cpp -o lrpot.so -shared -fPIC
\end{minted}
\end{tcolorbox}
具体可见:
\textcolor{blue}{\url{https://git.nju.edu.cn/bigd4/lrpot}}
\subsubsection{设置入口}
新建\file{magus}文件:
\begin{tcolorbox}[breakable, enhanced]
    \begin{minted}[breaklines, breaksymbolleft={},]{python}
import re
import sys
from magus.entrypoints.main import main
if __name__ == '__main__':
    sys.argv[0] = re.sub(r'(-script\.pyw|\.exe)?$', '', sys.argv[0])
    sys.exit(main())
    \end{minted}
\end{tcolorbox}
保存后执行\code{chmod +x magus}设置为可执行文件。
\subsubsection{设置环境变量}
在\file{bashrc}中加入
\begin{tcolorbox}[breakable, enhanced]
    \begin{minted}[breaklines, breaksymbolleft={},]{bash}
$ export PYTHONPATH=$PYTHONPATH:/your/path/magus
$ export PATH=$PATH:/your/path/magus
    \end{minted}
\end{tcolorbox}

\subsection{检查安装}
\begin{tcolorbox}
    \begin{minted}[breaklines, breaksymbolleft={},]{bash}
$ magus -v
    \end{minted}
\tcblower
1.0.2
\end{tcolorbox}

\subsection{设置自动补全(可选)}
在\file{bashrc}中加入
\begin{tcolorbox}[breakable, enhanced]
    \begin{minted}[breaklines, breaksymbolleft={},]{bash}
$ source your_path_to_magus/auto_complete.sh
    \end{minted}
\end{tcolorbox}
%%%%%%%%%%%%%%%%%%%%%%%%%%%%%%%%%%%%%%%%%%%%%%%%%
\newpage
\section{输入文件}
一个典型的结构搜索任务一般包含以下文件:
\begin{itemize}
    \item \file{input.yaml},给出任务的主要参数.详见 \ref{inputpara}
    \item \file{inputFold},补充\file{input.yaml}中定义的Calculator所需要的一些额外的信息,
    如Vasp的\file{INCAR},Lammps的\file{in.lammps}等.详见 \ref{inputfold}
    \item \file{Seeds},给出种子结构,可指定在哪一代加入.详见 \ref{seed}
\end{itemize}
\subsection{input.yaml参数} \label{inputpara}

\subsubsection{基本参数}
\begin{itemize}
    \item formulaType       : 计算类型,可用值:  fix(定组分),var(变组分)
    \item pressure          : 压强(GPa)
    \item molMode           : 是否使用分子晶体模式产生结构
    \item symprec           : 默认值:0.1, 判断空间群时容忍误差
\end{itemize}

\subsubsection{种群相关}
\begin{itemize}
    \item initSize          : 初代种群大小
    \item popSize           : 种群大小
    \item numGen            : 迭代次数
    \item saveGood          : 保留到下一代的优秀结构数
\end{itemize}

\subsubsection{结构产生}
\begin{itemize}
    \item spacegroup        : 随机结构的空间群, 
    例:[1,2,20-30]
    \item minNAtoms         : 最小原子数
    \item maxNAtoms         : 最大原子数
    \item symbols           : 元素类型
    例:['Ti', 'O'], 外层是方括号,每个元素用引号括起来, 
    \item formula           : 元素比例, 
    例:[1, 2] (定组分),[[1,0],[0,1]] (变组分)
    \item eleSize           : 变组分搜索时,产生的不包含所有元素的结构数, 
    如:eleSize=2, 则每代产生(He, Al, O, HexAly, AlxOy, HexOy)各两个
    \item volRatio          : 随机产生结构时的体积参数, 为结构体积除以原子球体积之和
    \item dRatio            : 若存在两原子距离除以共价半径之和小于此值, 则认为过近, 排除此结构
    \item distanceMatrix    : 原子间的最小距离矩阵, 与dRatio同时出现时使用此数值, 
    例:[[1, 1.5, 2.0], [1.5, 1.8, 1.9], [2.0, 1.9, 2.5]]
\end{itemize}

\subsubsection{结构演化}
\begin{itemize}
    \item randFrac          : 随机结构比例
    \item molDetector       : 结构演化时判断分子片段的方法
    可用值:0(不判断分子局域结构,默认值)  1(自动判断分子局域结构)  2(使用Girvan-Newman算法划分局域结构)
    \item addSym            : 产生结构之前是否为父代加入对称性
\end{itemize} 

\subsubsection{煮计算器}
主计算器定义了Magus搜索时使用的MainCalculator,以下所有参数都定义于MainCalculator条目下,需要缩进:
\begin{itemize}
    \item jobPrefix         : 计算器所需附属文件在\file{inputFold}文件夹中的名称,如\file{EMT}, \file{OvO}等
    如需使用多个计算器串接则给出对应列表,如:['Gulp', 'Vasp1', 'Vasp2']\label{jobprefix}
    \item calculator        : 程序种类,如未给出则按照jobPrefix文件名判断,可用值:
    {vasp, gulp, lammps, emt, xtb, lj, quip}
    \item mode              : 运行方式,可用值: serial (串行), parallel (并行)
    \item xc                : (VASP) 交换关联类型,可用值: PBE, LDA, PW-91
    \item ppLabel           : (VASP) VASP赝势的后缀,与symbols顺序一致,若无后缀则填入'',如:['\_sv', '']。
    \item exeCmd            : (gulp, lammps) 运行结构优化程序的命令,如gulp < input > output,mpirun -np 4 lmp\_mpi -in in.lammps
\end{itemize}
以下选项为并行模式下的队列控制选项,同样适用于代理计算器
\begin{itemize}
    \item numParallel       : 并行优化结构的数目
    \item numCore           : 结构优化使用的核数
    \item queueName         : 结构优化任务的队列
    \item verbose           : log中是否显示详细队列信息
    \item waitTime          : 检查任务的时间间隔(s)
    \item killTime          : 杀死任务的时间间隔(s)
    \item preProcessing     : 提交脚本时的预处理
\end{itemize}

\subsubsection{机器学习}
\begin{itemize}
    \item poolSize          : 用于初始化势场时随机产生的结构数
    \item initTimes         : 初始化势场时的迭代次数, 如果已有训练过的力场可设为0.默认值:2 
    \item DFTRelax          : 是否使用DFT演算
\end{itemize} 
\subsubsection{代理计算器}
代理计算器定义了MLMagus搜索时使用的MLCalculator,以下所有参数都定义于MLCalculator条目下,需要缩进:
\begin{itemize}
    \item jobPrefix: 同\ref{jobprefix}
    \item calculator: 程序种类,如未给出则按照jobPrefix文件名判断,可用值:
    {mtp, mtp-lammps}
    \item connect: 多个mtp的连接方式,可用值:[naive, share-trainset]
    \item force\_tolerance: 结构优化力收敛判据,默认值:0.05
    \item stress\_tolerance: 结构优化应力收敛判据,默认值:1.
    \item weights: 训练时能量、力、应力权重,默认值:[1., 0.01, 0.001]
    \item scaled\_by\_force:训练时给予较小力的额外权重,默认值:0.
    \item min\_dist: 优化时最小距离, 默认值:0.5
    \item n\_epoch:训练代数,默认值:200
\end{itemize}

\subsection{inputFold} \label{inputfold}
\file{inputFold}中为不同calculator所需要的补充文件,放在对应的\file{jobPrefix}文件夹中。
以下为各calculator所需文件的示例,内容可根据需要修改(名字不行):
\subsubsection{Vasp}
\begin{tcolorbox}[enhanced, breakable, title = {INCAR}]
就是INCAR
\tcblower
\begin{verbatim}                                
PREC = Accurate
EDIFF = 1e-4
EDIFFG = 1e-3
IBRION = 2
ISIF = 3
NSW = 40
ISMEAR = 0
SIGMA = 0.050
POTIM = 0.250
ISTART = 0
LCHARG = FALSE
LWAVE = FALSE
KSPACING = 0.314
NCORE= 4
\end{verbatim}
\end{tcolorbox}
注意:\file{INCAR}中不需要给出pstress
%%%%%%%%%%%%%%%%%%%%%%%%%%%%%%%%%%%%%%%%%%
% Gulp输入文件
\subsubsection{Gulp}
% relax文件
\begin{tcolorbox}[enhanced, breakable, title = {goption.relax}]
gulp优化参数配置
\tcblower
\begin{verbatim}                                
    opti conjugate nosymmetry conp
\end{verbatim}
\end{tcolorbox}
% scf文件
\begin{tcolorbox}[enhanced, breakable, title = {goption.scf}]
gulp自洽参数配置
\tcblower
\begin{verbatim}                                
    nosymmetry conp gradients
\end{verbatim}
\end{tcolorbox}
% 势函数
\begin{tcolorbox}[enhanced, breakable, title = {goption.scf}]
gulp势函数文件
\tcblower
\begin{minted}[breaklines, breaksymbolleft={},]{text}                                
space                                                                                                         
1
species
Mg  2.0 
Al  3.0 
O  -2.0
lennard 12 6
Mg O   1.50 0.00 0.00 6.0 
Al O   1.50 0.00 0.00 6.0 
O O    1.50 0.00 0.00 6.0 
Mg Mg  1.50 0.00 0.00 6.0 
Mg Al  1.50 0.00 0.00 6.0 
Al Al  1.50 0.00 0.00 6.0 
buck
Mg O 1428.5 0.2945 0.0 0.0 7.0 
Al O 1114.9 0.3118 0.0 0.0 7.0 
O O  2023.8 0.2674 0.0 0.0 7.0 
maxcyc 850 
switch rfo 0.010
time 60
\end{minted}
\end{tcolorbox}
%%%%%%%%%%%%%%%%%%%%%%%%%%%%%%%%%%%%%%%%%%
% Lammps输入文件
\subsubsection{Lammps}
% relax文件
\begin{tcolorbox}[enhanced, breakable,title = {in.relax}]
lammps优化参数配置,可替换为分子动力学等等
\tcblower
\begin{minted}[breaklines, breaksymbolleft={},]{text}                                
clear
atom_style atomic 
units metal
boundary p p p 
read_data data            #此行不可更改
### interactions
pair_style lj/cut 2.5 
pair_coeff * * 1 1 
mass 1 35.450000 
mass 2 22.989769 
### run 
fix fix_nve all nvt temp 300.0 300.0 100 
dump dump_all all custom 1 out.dump id type x y z vx vy vz fx fy fz
# 最终输出必须为 out.dump
thermo_style custom step temp press pxx pyy pzz pxy pxz pyz ke pe etotal
# 需要pxx pyy pzz pxy pxz pyz
thermo 1
run 10
\end{minted}
\end{tcolorbox}
% scf文件
\begin{tcolorbox}[enhanced, breakable, title = {in.scf}]
lammps自洽计算参数配置
\tcblower
\begin{minted}[breaklines, breaksymbolleft={},]{text}                               
clear
atom_style atomic 
units metal
boundary p p p 
read_data data
### interactions
pair_style lj/cut 2.5 
pair_coeff * * 1 1 
mass 1 35.450000 
mass 2 22.989769 
### run 
fix fix_nve all nve 
dump dump_all all custom 1 out.dump id type x y z vx vy vz fx fy fz
thermo_style custom step temp press pxx pyy pzz pxy pxz pyz ke pe etotal
thermo 1
run 10
\end{minted}
\end{tcolorbox}
%%%%%%%%%%%%%%%%%%%%%%%%%%%%%%%%%%%%%%%%%%
% ASE输入文件
\subsubsection{ASE系列 (EMT, LJ, XTB\dots)}
\begin{tcolorbox}[enhanced, breakable, title = {EMT, LJ, XTB\dots}]
建个文件夹就完事了,除非如XTB需要相关配置文件
\tcblower
xtb.yaml \\下次一定
\end{tcolorbox}
%%%%%%%%%%%%%%%%%%%%%%%%%%%%%%%%%%%%%%%%%%
% MTP输入文件
% mlip.ini
\subsubsection{MTP}
\begin{tcolorbox}[enhanced, breakable, title = {mlip.ini}]
active learning控制参数,详见
\textcolor{blue}{\url{https://git.nju.edu.cn/bigd4/mtp-api/-/blob/master/doc/manual/manual.pdf}}
\tcblower
\begin{minted}[breaklines, breaksymbolleft={},]{text} 
mtp-filename                        pot.mtp                 #改不得                                                                 
select                              TRUE                  
    select:site-en-weight           1.0     
    select:energy-weight            0.0   
    select:force-weight             0.0  
    select:stress-weight            0.0 
    select:threshold                1.5 
    select:threshold-break          7.0 
    select:save-selected            B-preselected.cfg       #改不得
    select:load-state               A-state.als             #改不得
\end{minted}
\end{tcolorbox}
%pot.mtp
\begin{tcolorbox}[enhanced, breakable, title = {pot.mtp}]
mtp使用势场,可从untrained\_mtps中拷贝,仅可改变标注了作用的参数。
\tcblower
\begin{minted}[breaklines, breaksymbolleft={},]{text} 
MTP                                                                                                           
version = 1.1.0
potential_name = MTP1m
species_count = 1                 # 原子种类数量
potential_tag = 
radial_basis_type = RBChebyshev
    min_dist = 2                  # 原子最小间距
    max_dist = 5                  # 原子环境最大考虑半径
    radial_basis_size = 8         # 基函数个数
    radial_funcs_count = 3 
alpha_moments_count = 84
alpha_index_basic_count = 46     
\end{minted}
\end{tcolorbox}
%train.cfg
\begin{tcolorbox}[enhanced, breakable, title = {train.cfg}]
训练集,若不存在,将自动生成空训练集
\tcblower
\begin{minted}[breaklines, breaksymbolleft={},]{text} 
BEGIN_CFG                                                                                                     
Size
4
Supercell
2.457244 0.0 0.0
0.0 -2.457244 0.0
0.0 0.0 -2.457244
AtomData: id type cartes_x cartes_y cartes_z fx fy fz 
1 0 0.0 0.0 0.0 0.0 -0.0 0.0 
2 0 0.0 -1.228622 -1.228622 0.0 -0.0 -0.0  
3 0 1.228622 -1.228622 0.0 -0.0 0.0 0.0 
4 0 1.228622 0.0 -1.228622 0.0 -0.0 -0.0  
Energy
-15.79300182
EnergyWeight
0.02195049074783156
PlusStress: xx yy zz yz xz xy 
26.253439887628005 26.253439887628005 26.253439887628005 -0.0 0.0 0.0
END_CFG    
\end{minted}
\end{tcolorbox}
\subsection{Seeds} \label{seed}
%POSCARS_i
POSCARS\_i为第i代加入的种子文件,如有\file{POSCAR\_1}$\sim$\file{POSCAR\_9}希望在第一代加入,
\file{POSCAR\_10}$\sim$\file{POSCAR\_19}希望第二代加入,执行\code{cat POSCAR\_\{1..9\} > POSCARS\_1};
\code{cat POSCAR\_\{10..19\} > POSCARS\_2}后将\file{POSCARS\_1}与\file{POSCARS\_2}放入\file{Seeds}中即可。

%%%%%%%%%%%%%%%%%%%%%%%%%%%%%%%%%%%%%%%%%%%%%%%%%
\newpage
\section{程序指令}
\file{magus}文件为程序运行的入口,通过运行\file{magus}可以得到所有的指令与介绍,
通过 \code{magus [commond] -h} 可获得其帮助。
\begin{center}
\begin{tabular}{p{100pt}<{\centering} p{200pt}<{\centering}}
    \toprule
    指令                   & 用途 \\
    \midrule 
    search~\ref{search}    & 结构搜索 \\
    summary~\ref{summary}  & 事后总结 \\
    clean                  & 事后清理 \\
    prepare                & 事前准备 \\
    calc~\ref{calc}        & 批量计算 \\
    gen~                   & 批量产生 \\
    \bottomrule
\end{tabular}
\end{center}
\subsection{search} \label{search}
结构搜索模块,使用时直接提交命令\code{magus search}即可。可选择参数如下:
\begin{itemize}
    \item -h --help \\
    展示帮助文档
    \item -i INPUTFILE, --input-file INPUTFILE \\
    指定输入参数文件,默认为\file{input.yaml}
    \item -l LEVEL, --log-level LEVEL\\
    指定log.txt文件logging等级,可选项:DEBUG,INFO,WARNING,ERROR。默认为INFO
    \item -m, --use-ml   \\ 
    是否使用机器学习模块
    \item -r, --restart \\   
    是否为继续上次任务,使用此选项目录内应保留上次作业的\file{results}与\file{log.txt}
\end{itemize}
\subsection{summary} \label{summary}
用于总结一条或多条ase traj格式的轨迹,参数如下:
\begin{itemize}
    \item -h --help \\
    展示帮助文档
    \item -p PREC, --prec PREC  \\
    判断空间群的精度,默认值为0.1
    \item -r, --reverse     \\    
    是否倒着输出,默认正着输出
    \item -s, --save \\ 
    是否将此轨迹中所有结构输出为POSCAR,默认不输出,以防文件夹很乱
    \item -o OUTDIR, --outdir OUTDIR\\
    POSCAR输出的目录
    \item  -n SHOW\_NUMBER, --show-number SHOW\_NUMBER \\
    展示条目数量,默认为20
    \item  -sb SORTED\_BY, --sorted-by SORTED\_BY\\
    用哪个关键字排序,默认为enthalpy
    \item -rm REMOVE\_FEATURES [REMOVE\_FEATURES ...], 
    --remove-features REMOVE\_FEATURES [REMOVE\_FEATURES ...]\\
    需要移除展示的信息
    \item -a ADD\_FEATURES [ADD\_FEATURES ...], 
    --add-features ADD\_FEATURES [ADD\_FEATURES ...]\\
    需要附加展示的其他信息
\end{itemize}
一个例子如下:
\begin{tcolorbox}[enhanced, breakable]
    \begin{minted}[breaklines, breaksymbolleft={},]{bash} 
$ magus summary good.traj ref.traj -a volume -rm priSym parentE -n 10
    \end{minted}
\tcblower
\begin{minted}[breaklines, breaksymbolleft={},]{text} 
        symmetry  enthalpy  origin fullSym   volume source
1   I-43d (220)     0.412   random    Li16  123.164    ref
2   Cmc2_1 (36)     0.416     seed    Li88  647.965   good
3     Aea2 (41)     0.419     None    Li40  301.519    ref
4     Ama2 (40)     0.419   random    Li88  645.352   good
5    P3c1 (158)     0.422   Rattle    Li88  644.677   good
6        Cc (9)     0.423   Rattle    Li88  646.661   good
7        Cm (8)     0.427  Lattice    Li88  649.852   good
8        P1 (1)     0.427     Slip    Li88  647.470   good
9     Aea2 (41)     0.429   random    Li88  647.528   good
10    Aea2 (41)     0.429   Rattle    Li88  647.652   good
\end{minted} 
\end{tcolorbox}

\subsection{calc} \label{calc}
根据\file{input.yaml}中定义的calculator,计算给出的traj,结果会输出于\file{out.traj},
可用\code{magus summary out.traj}命令查看。
\begin{itemize}
    \item -h --help \\
    展示帮助文档
    \item -m MODE, --mode MODE \\
    计算类型,可用值:scf, relax. 默认为relax
    \item -i INPUTFILE, --input-file INPUTFILE  \\    
    指定参数所在文件,默认为\file{input.yaml}
    \item -p PRESSURE, --pressure PRESSURE \\ 
    指定压强,若不给出则使用INPUTFILE中给出的压强
\end{itemize}
如若需对\file{in.traj}中的结构在10GPa下优化,运行命令为:
\begin{tcolorbox}
    \begin{minted}[breaklines, breaksymbolleft={},]{bash} 
$ magus calc in.traj -p 10
    \end{minted}
\end{tcolorbox}
此命令与MtpCalculator或MTPLammpsCalculator结合,可以on the fly的得到机器学习的训练集。
\newpage
\section{输出文件}
结构搜索过程中,将会产生\file{log.txt},\file{calcFold}与\file{results}, 如果搜索时加入-m选项,
会额外产生\file{MLFold}。
\subsection{calcFold}
\file{calcFold}中为\file{input.yaml}定义的计算器计算过程中所产生的文件。
一个典型的结构搜索任务将产生如下结构的\file{calcFold}:

\begin{verbatim}
        calcFold
        ├── MTP
        │   ├── epoch00
        │   ├── epoch01
        │   ├── epoch02
        │   └── job_controller
        └── Vasp
            ├── 00
            ├── 01
            ├── 02
            ├── 03
            └── job_controller
\end{verbatim}
\par
如果并行模式计算过程中发生错误,报错信息将在对应的文件夹中出现。
此外,并行模式中计算器文件夹下会产生\file{job\_controller}文件,
可通过修改其中内容改变作业提交的参数。如:

\begin{center}
    \begin{tabular}{p{200pt}<{\centering} p{200pt}<{\centering}}
        \begin{tcolorbox}[title=old\_job\_controller]
            \begin{minted}[breaklines, breaksymbolleft={},]{yaml} 
kill_time: 100000                                                             
num_core: 48
pre_processing: ''
queue_name: 9242opa!
verbose: false
            \end{minted}
        \end{tcolorbox}                   
        & 
        \begin{tcolorbox}[title=new\_job\_controller]
            \begin{minted}[breaklines, breaksymbolleft={},]{yaml} 
kill_time: 100000                                                              
num_core: 64
pre_processing: ''
queue_name: 7702ib
verbose: false
            \end{minted}
        \end{tcolorbox}
    \end{tabular}
\end{center}
\par
这些改变将被记录在\file{log.txt}中:
\begin{tcolorbox}[title=log.txt]
    \begin{minted}[breaklines, breaksymbolleft={},]{text} 
Be careful, the following settings are changed                                                
    num_core: 48 -> 64
    queue_name: 9242opa! -> 7702ib
    \end{minted}
\end{tcolorbox}                   

\subsection{mlFold}
\file{mlFold}中为\file{input.yaml}中定义的机器学习模块执行挑选、训练的部分。
如果提交任务时此文件夹不存在,将会使用\file{inputFold}中对应的文件。
训练或挑选中发生错误报错信息将在对应文件夹中出现,此外,可在\file{train-out}中查看训练集上的误差。
\par
若已经进行过一次MLMagus搜索并得到了一个不错的势,可以不用删除\file{mlFold}以在以后的搜索中使用;
或者复制\file{pot.mtp}与\file{train.cfg}到其他\file{inputFold}中反复使用。
此时可将init\_times设为0,代表不再在初始化时训练力场。
\subsection{results}
\file{results}中记录了各代生成的各种结构,可根据需要使用summary命令查看。
\begin{itemize}
    \item \file{good.traj}\\
    目前最优的popSize个结构
    \item \file{good\{i\}.traj}\\
    第i代最优的popSize个结构
    \item \file{best.traj}\\
    历代最优的结构,使用summary查看时注意需添加-sb None或--sorted\_by None选项,
    否则会默认按焓值排序显示而不是代数的顺序
    \item \file{keep\{i\}.traj}\\
    第i代经过聚类后保留的goodSize个结构
    \item \file{init\{i\}.traj}\\
    第i代产生的初始结构
    \item \file{raw\{i\}.traj}\\
    第i代的初始结构经过第一性结构优化后的结构,可用于debug
    \item \file{gen\{i\}.traj}\\
    第i代种群,一般为raw\{i\}.traj或mlraw\{i\}.traj经过check后的结果
    若机器学习搜索使用第一性验证,则为mlraw\{i\}.traj中低能结构第一性优化后额结果。
    第一性验证则为
    \item \file{mlraw\{i\}.traj}\\
    第i代的初始结构经过机器学习结构优化后的结构,可用于debug
    \item \file{mlgen\{i\}.traj}\\
    mlraw\{i\}.traj经过check的结果
\end{itemize}
\newpage
\section{例子}
\subsection{\texorpdfstring{$NH_4NO_3$分子晶体产生}{分子晶体产生}} \label{molgen}
目标:
产生10个$Pccn$的$NH_4NO_3$分子晶体。\par
准备:\file{input.yaml}, \file{NH4.xyz}, \file{NO3.xyz}
% input.yaml
\begin{tcolorbox}[enhanced, breakable, title = {\file{input.yaml}}]
结构产生控制文件
\tcblower
\begin{minted}[breaklines, breaksymbolleft={},]{yaml}
    minAt: 72                        # 最小原子数
    maxAt: 72                        # 最大原子数
    symbols: ['H', 'N', 'O']         # 待产生结构所含元素
    molMode: True                    # 分子晶体模式
    molFile: ['NH4.xyz', 'NO3.xyz']  # 所含分子结构
    molFormula: [1, 1]               # 分子组分
    molType: 'fix'
    spacegroup: [56]                 # 指定56号空间群Pccn
    dRatio: 0.8                      # 原子间最小距离比
    threshold_mol: 1.5               # 分子间最小距离比
    volRatio: 8                      # 体积比
\end{minted}
\end{tcolorbox}
% NH4.xyz
\begin{tcolorbox}[enhanced, breakable, title = {\file{NH4.xyz}}]
    \begin{minted}[breaklines, breaksymbolleft={},]{text} 
    5                                                                                 
    NH4
    H    4.511281    4.375470    3.210227
    H    3.584655    4.486488    1.796710
    H    4.670180    3.191076    2.019142
    H    3.246077    3.272899    2.937012
    N    4.000271    3.837356    2.488938
    \end{minted}
\end{tcolorbox}
% NO3.xyz
\begin{tcolorbox}[enhanced, breakable, title = {\file{NO3.xyz}}]
    \begin{minted}[breaklines, breaksymbolleft={},]{text} 
    4                                                                                 
    NO3
    N    2.012707    2.014563    4.870574
    O    1.714319    0.953807    5.478185
    O    2.311095    3.075319    5.478185
    O    2.012707    2.014563    3.582428
\end{minted}
\end{tcolorbox}
运行:\code{magus gen -n 10} \par
结果:目标结构\file{gen.traj}
\subsection{MTP大批量优化随机结构} \label{mtpB}
目标:使用MTP力场优化1000个$B_{12}$随机结构\par
准备:待优化结构\file{gen.traj},\file{inputFold},\file{input.yaml},
其中,inputFold结构为:
\begin{verbatim}
        inputFold/
            ├── MTP
            │   ├── mlip.ini
            │   ├── pot.mtp
            │   └── train.cfg
            └── Vasp
                └── INCAR
\end{verbatim}

% input.yaml
\begin{tcolorbox}[enhanced, breakable, title = {\file{input.yaml}}]
定义主计算器与机器学习计算器
\tcblower
\begin{minted}[breaklines, breaksymbolleft={},]{yaml}
#main calculator settings
    MainCalculator:
        jobPrefix: Vasp          # 标准能量由Vasp给出
        #vasp settings
        xc: PBE 
        ppLabel: ['']
        #parallel settings
        numParallel: 20
        numCore: 24
        queueName: 9242opa!
        waitTime: 30

    MLCalculator:
        jobPrefix: MTP
        calculator: mtp
        min_dist: 1.2           # MTPrelax最小距离
        queueName: 9242opa!
        numCore: 48
        waitTime: 10
        force_tolerance: 0.001 
        stress_tolerance: 0.01
\end{minted}
\end{tcolorbox}
% INCAR
\begin{tcolorbox}[enhanced, breakable, title = {\file{INCAR}}]
    不会有人不会写INCAR把
    \tcblower
    \begin{minted}[breaklines, breaksymbolleft={},]{text}                                                     
    PREC = Accurate
    EDIFF = 1e-4
    IBRION = 2 
    ISIF = 3 
    NSW = 40
    ISMEAR = 0 
    SIGMA = 0.050
    POTIM = 0.250
    ISTART = 0 
    LCHARG = FALSE
    LWAVE = FALSE
    #Crude optimisation
    EDIFFG = 1e-3
    KSPACING = 0.314
    NCORE = 4
    \end{minted}
\end{tcolorbox}
% pot.mtp
\begin{tcolorbox}[enhanced, breakable, title = {\file{pot.mtp}}]
    MTP势函数文件,这里只给出表头
    \tcblower
    \begin{minted}[breaklines, breaksymbolleft={},]{text}
    MTP                                                                                   
    version = 1.1.0
    potential_name = MTP1m
    scaling = 1.497018669914417e-02
    species_count = 1                     # 只有B原子一种
    potential_tag = 
    radial_basis_type = RBChebyshev
        min_dist = 1.277113860000000e+00  
        # 最小距离1.27,由于train中设置了--update,此项不必特别设置,会自动更新
        max_dist = 6.000000000000000e+00
        radial_basis_size = 12
        radial_funcs_count = 5
    \end{minted}
\end{tcolorbox}
% mlip.ini
\begin{tcolorbox}[enhanced, breakable, title = {\file{mlip.ini}}]
    active控制文件
    \tcblower
    \begin{minted}[breaklines, breaksymbolleft={},]{text}
    mtp-filename                    pot.mtp                                               
    select                          TRUE               
        select:site-en-weight           1.0                  
        select:energy-weight            0.0                  
        select:force-weight             0.0                
        select:stress-weight            0.0                 
        select:threshold                1.5                
        select:threshold-break          7.0               
        select:save-selected            B-preselected.cfg   
        select:load-state               A-state.als        
    \end{minted}
\end{tcolorbox}
运行:提交\code{magus calc gen.traj}命令到队列\par
结果:MTP优化后的结构\file{out.traj},
势文件\file{mlFold/MTP/pot.mtp}
\subsection{\texorpdfstring{$TiO_2$定组分结构搜索}{TiO2定组分结构搜索}} \label{fixsearch}
目标:搜索12个原子$TiO_2$的结构\par
准备:\file{input.yaml}, \file{inputFold}\par
由于初始结构往往杂乱无章,因此往往使用多个\file{INCAR}分步优化,
据说合适的做法是固定体积优化原子位置与晶格形状(ISIF=4),
然后放开体积限制优化原子位置与晶格(ISIF=3), 
最后进行高精度的单点能自洽运算(NSW=0)
% input.yaml
\begin{tcolorbox}[enhanced, breakable, title = {\file{input.yaml}}]
    给出搜索所需参数
    \tcblower
    \begin{minted}[breaklines, breaksymbolleft={},]{yaml}
    calcType: fix                 # 定组分搜索
    pressure: 0                   # 0GPa
    initSize: 20                  # 初始生成20个结构
    popSize: 20                   # 每代维持20个结构
    numGen: 10                    # 搜索10代                                                                                      
    saveGood: 3                   # 每代保存3个结构

    #structure parameters
    minAt: 12                     
    maxAt: 12
    symbols: ['Ti', 'O']
    formula: [1, 2]

    dRatio: 0.6                   # 最小半径比
    volRatio: 3                   # 最小体积比
    addSym: False                 # 不在父代中加入对称性

    #main calculator settings
    MainCalculator:
        jobPrefix: ['q', 'w', 'e' ,'r']
        # 这里只是为了说明只要给出calculator,jobPrefix可以随意命名,实际建议用更清晰的方式命名。
        calculator: vasp
        mode: parallel
        #vasp settings
        xc: PBE 
        ppLabel: ['', '_s']
        #parallel settings
        numParallel: 5
        queueName: 7702ib
        waitTime: 30
    \end{minted}
    \end{tcolorbox}

\begin{paracol}{2}
    \begin{tcolorbox}[enhanced, breakable, title=\file{q/INCAR}]
        \begin{minted}[breaklines, breaksymbolleft={},]{text} 
PREC = Normal
EDIFF = 1e-3
IBRION = 2 
ISIF = 4 
NSW = 85
ISMEAR = 0
SIGMA = 0.06
POTIM = 0.20
LCHARG = FALSE
LWAVE = FALSE
#Crude optimisation
EDIFFG = 1e-2
KSPACING = 1.256
LREAL = A 
ALGO = Fast
        \end{minted}
    \end{tcolorbox}                   
    %
    \begin{tcolorbox}[enhanced, breakable, title=\file{e/INCAR}]
        \begin{minted}[breaklines, breaksymbolleft={},]{text} 
PREC = Normal
EDIFF = 1e-4
IBRION = 2 
ISIF = 3 
NSW = 70
ISMEAR = 0
SIGMA = 0.060
POTIM = 0.250
ISTART = 0 
LCHARG = FALSE
LWAVE = FALSE
#Crude optimisation
EDIFFG = 1e-3
KSPACING = 0.618
#ALGO = Fast
        \end{minted}
    \end{tcolorbox}  
    \switchcolumn[1]
    \begin{tcolorbox}[enhanced, breakable, title=\file{w/INCAR}]
        \begin{minted}[breaklines, breaksymbolleft={},]{text} 
PREC = Normal
EDIFF = 1e-3
IBRION = 2 
ISIF = 3 
NSW = 100 
ISMEAR = 0
SIGMA = 0.060
POTIM = 0.020
LCHARG = FALSE
LWAVE = FALSE
#Crude optimisation
EDIFFG = 1e-2
KSPACING = 0.942
LREAL = A 
ALGO = Fast
        \end{minted}
    \end{tcolorbox}           
    %
    \begin{tcolorbox}[enhanced, breakable, title=\file{r/INCAR}]
        \begin{minted}[breaklines, breaksymbolleft={},]{text} 
PREC = Normal
EDIFF = 1e-4
IBRION = 2 
ISIF = 2 
NSW = 0 
ISMEAR = 0 
SIGMA = 0.060
POTIM = 0.250
ISTART = 0 
LCHARG = FALSE
LWAVE = FALSE
#Crude optimisation
EDIFFG = 1e-3
KSPACING = 0.618
#ALGO = Fast
        \end{minted}
    \end{tcolorbox}
\end{paracol}
运行:提交\code{magus search}到队列\par
结果:搜索结果\file{results/good.traj}
%%%%%%%%%%%%%%%%%%%%%%%%%%%%%%%%%
\subsection{\texorpdfstring{$Zn_x(OH)_y$变组分结构搜索}{ZnOH变组分结构搜索}}
目标:搜索8-16个原子$Zn_x(OH)_y$的结构\par
准备:\file{input.yaml}, \file{inputFold}\par
使用Gulp经验势优化,因此种群数目与代数可以大大增加。
% input.yaml
\begin{tcolorbox}[enhanced, breakable, title = {\file{input.yaml}}]
    给出搜索所需参数
    \tcblower
    \begin{minted}[breaklines, breaksymbolleft={},]{yaml}
    calcType: var               # 变组分搜索
    pressure: 0
    initSize: 150 
    popSize: 150 
    numGen: 60
    saveGood: 8
    #structure parameters
    minAt: 8
    maxAt: 16
    symbols: ['Zn','O','H']
    formula: [[1,0,0],[0,1,1]]  # Zn : (OH) = 1 : 1
    fullEles: True
    eleSize: 5
    dRatio: 0.5 
    volRatio: 10
    addSym: False
    #main calculator settings
    MainCalculator:
        jobPrefix: ['Gulp1', 'Gulp2']  
        # 若jobPrefix给出计算器名称,可不指定calculator                                                 
        mode: parallel
        #gulp settings
        exeCmd: gulp < input > output
        #parallel settings
        numParallel: 5
        numCore: 4
        queueName: e52692v2ib!
        waitTime: 30
    \end{minted}
    \end{tcolorbox}

\begin{tcolorbox}[enhanced, breakable, title=\file{Gulp1/gpot}]
定义gulp所使用的势,其中\file{ReaxFF.lib}是相应的反应力场文件
\tcblower
\begin{minted}[breaklines, breaksymbolleft={},]{text} 
    time 240                                                                                                                    
    space
    1
    maxcyc 300 
    library ./ReaxFF.lib
    lennard epsilon
    Zn  Zn  0.0150 1.00 0.0 8.0 
    Zn  O   0.0150 1.00 0.0 8.0 
    Zn  H   0.0150 0.80 0.0 8.0 
    H  H    0.0150 0.60 0.0 8.0 
    O  O    0.0150 0.80 0.0 8.0 
    H  O    0.0150 0.80 0.0 8.0
\end{minted}
\end{tcolorbox}
\begin{tcolorbox}[enhanced, breakable, title=\file{Gulp1/goption.relax}]
relax所使用命令,conv代表第一代粗优固定晶格优化
\tcblower
\begin{minted}[breaklines, breaksymbolleft={},]{text} 
    opti spatial conj nosymmetry conv
\end{minted}
\end{tcolorbox}   
%
类似的:
\begin{tcolorbox}[enhanced, breakable, title=\file{Gulp2/gpot}]
\begin{minted}[breaklines, breaksymbolleft={},]{text} 
    time 240                                                                                                                    
    space
    1
    maxcyc 300 
    library ./ReaxFF.lib
\end{minted}
\end{tcolorbox}
\begin{tcolorbox}[enhanced, breakable, title=\file{Gulp1/goption.relax}]
\begin{minted}[breaklines, breaksymbolleft={},]{text} 
    opti spatial conj nosymmetry conv
\end{minted}
\end{tcolorbox}
运行:提交\code{magus search}到队列\par
结果:搜索结果\file{results/good.traj}
\subsection{\texorpdfstring{$MgSiO_3$机器学习结构搜索}{MgSiO3机器学习结构搜索}}
目标:使用MTP加速搜索10-20个原子$MgSiO_3$的结构\par
准备:\file{input.yaml}, \file{inputFold}\par
与\ref{mtpB}类似,准备好相应的\file{pot.mtp}, \file{mlip.ini}, 
不同的是需要把\file{pot.mtp}中的原子种类改为3种
% input.yaml
\begin{tcolorbox}[enhanced, breakable, title = {\file{input.yaml}}]
    搜索所需参数,计算器部分与\ref{mtpB}类似,不再赘述
    \tcblower
    \begin{minted}[breaklines, breaksymbolleft={},]{yaml}
    calcType: fix                                                              
    poolSize: 2000    # 预训练生成的代挑选随机结构,可以设的很大
    initSize: 400
    popSize: 400
    numGen: 60
    saveGood: 3

    #structure parameters
    DFTRelax: False   # 不使用DFT验证能量
    minAt: 10
    maxAt: 20
    symbols: ['Mg','Si','O']
    formula: [1, 1, 3]
    molDetector: 2    # 2号分子探测算法
    dRatio: 0.8 
    volRatio: 1.3 
    randFrac: 0.4 
    pressure: 150 
    addSym: True
    softNum: 0
    \end{minted}
\end{tcolorbox}
%
\begin{tcolorbox}[enhanced, breakable, title=\file{MTP/pot.mtp}]
    表头部分,主要区别为分子种类被替换成了3
    \tcblower
\begin{minted}[breaklines, breaksymbolleft={},]{text} 
    MTP                                                                        
    version = 1.1.0
    potential_name = MTP1m
    scaling = 7.002314814814817e-01
    species_count = 3 
    potential_tag = 
    radial_basis_type = RBChebyshev
        min_dist = 5.000000000000000e-01
        max_dist = 5.000000000000000e+00
        radial_basis_size = 12
        radial_funcs_count = 4
\end{minted}
\end{tcolorbox}
运行:提交\code{magus search -m}到队列\par
结果:搜索结果\file{results/good.traj}
\subsection{\texorpdfstring{$CH_4$分子晶体搜索}{CH4分子晶体搜索}}
目标:甲烷晶体结构搜索
准备:\file{input.yaml}, \file{inputFold}, \file{CH4.xyz}\par
\file{input.yaml}可参照\ref{molgen}与\ref{fixsearch}中的设置:
\begin{tcolorbox}[enhanced, breakable, title=\file{MTP/pot.mtp}]
\begin{minted}[breaklines, breaksymbolleft={},]{yaml} 
    calcType: fix                                                              
    initSize: 20
    popSize: 20
    numGen: 40
    saveGood: 3
    pressure: 50
    minAt: 20
    maxAt: 20
    symbols: ['C', 'H']
    ## mol crystal
    molMode: True
    molFile: ['CH4.xyz']
    molFormula: [4] 
    molType: 'fix'
    chkMol: True
    addSym: True
    
    dRatio: 0.8 
    volRatio: 5
    randFrac: 0.4 
    molDetector: 2
    
    #main calculator settings
    MainCalculator:
        jobPrefix: ['Vasp1', 'Vasp2']
        mode: parallel
        #vasp settings
        xc: PBE 
        ppLabel: ['','']
        #parallel settings
        numParallel: 4
        numCore: 24
        queueName: 9242opa!
\end{minted}
\end{tcolorbox}
\begin{tcolorbox}[enhanced, breakable, title=\file{MTP/pot.mtp}]
    \begin{minted}[breaklines, breaksymbolleft={},]{text} 
    5                                                                          
    C  H   
    C    2.260984    1.227715    2.255654
    H    2.597307    0.217093    2.238728
    H    1.194544    1.227505    2.236584
    H    2.611534    1.725297    1.379429
    H    2.590593    1.698611    3.156207
    \end{minted}
\end{tcolorbox}
运行:提交\code{magus search} 到队列\par
结果:搜索结果\file{results/good.traj}
\newpage
\section{常见问题}
使用时遇到疑问或bug可在\textcolor{blue}{\url{https://git.nju.edu.cn/gaaooh/magus}}中提出issue.
\begin{enumerate}
    \item \textbf{为啥pip安装时报错"ModuleNotFoundError: No module named 'yaml'"?}
    \par
    那你装一个啊
    \begin{tcolorbox}[enhanced, breakable]
        \begin{minted}[breaklines, breaksymbolleft={},]{bash} 
$ pip install pyyaml
        \end{minted}
    \end{tcolorbox}
    
\end{enumerate}

\end{document}
